%
%  Time-stamp: "[abstract.tex] last modified by Scott Budge (scott) on 2017-01-10 (Tuesday, 10 January 2017) at 16:54:14 on goga.ece.usu.edu"
%
%  Info: $Id: abstract.tex 998 2017-03-21 16:44:33Z scott $   USU
%  Revision: $Rev: 998 $
% $LastChangedDate: 2017-03-21 10:44:33 -0600 (Tue, 21 Mar 2017) $
% $LastChangedBy: scott $
%

\begin{abstract}
% A space is needed before the text starts so that the first paragraph
% is indented properly.

This thesis is designed to generate a computational model that simulates the process of granulation in anaerobic sludge and aims to investigate scenarios of possible granular bioaugmentation. Bioaugmentation is a common strategy in the field of wastewater treatment, used to introduce a new metabolic capability to either aerobic or anaerobic granules. The end product is a model that can visually demonstrate varying stratifications of different trophic microbial groups that will be of help for the engineers and researchers, who are operating both laboratory and industrial-scale anaerobic digesters and wish to enhance reactor performance. The working model that we have developed has been validated using the existing literature and lab experiments. The model successfully demonstrates granulation and bioaugmentation a cellobiose and/or oleate fed system.

\end{abstract}


% Local Variables:
% TeX-master: "newhead"
% End:
