%
%  Example Appendix pages.
%  Modified to use new usu-thesis-mk2 appendix facilities.
%
%  Time-stamp: "[appendix.tex] last modified by Scott Budge (scott) on 2011-08-08 (Monday, 8 August 2011) at 15:46:06 on goga"
%
%  Info: $Id$   USU
%  Revision: $Rev$
% $LastChangedDate$
% $LastChangedBy$
%
%
% For a single appendix, use \makeappendix, and place the 
% body of the appendix after it

%\makeappendix

% < single appendix body here >

% For multiple appendices, use \makeappendices, and create each appendix
% using \appendix{}
% For sub-appendices use \appendixsection{} and \appendixsubsection{}

\makeappendices
\appendix{Input xml and result analysis code}
\label{chap:appendix}


\appendixsection{Description}

This xml document function as the definition of the model and as the input to the
simulation framework.


\label{sec:edge-def}

\appendixsection{Protocol file to generate the complex granule on cellobiose}

    \lstset{
    language=xml,
    tabsize=3,
    %frame=lines,
    caption=Input xml,
    label=code:sample,
    frame=shadowbox,
    rulesepcolor=\color{gray},
    xleftmargin=20pt,
    framexleftmargin=15pt,
    keywordstyle=\color{blue}\bf,
    commentstyle=\color{OliveGreen},
    stringstyle=\color{red},
    numbers=left,
    numberstyle=\tiny,
    numbersep=5pt,
    breaklines=true,
    showstringspaces=false,
    basicstyle=\footnotesize,
    emph={food,name,price},emphstyle={\color{magenta}}}
    \lstinputlisting{new2.xml}
    
    \appendixsection{Result analysis code to generate spatial distribution data}
    \lstset{frame=tb,
  language=Java,
  aboveskip=3mm,
  belowskip=3mm,
  showstringspaces=false,
  columns=flexible,
  caption=Species biomass calculator,
  basicstyle={\small\ttfamily},
  numbers=none,
  numberstyle=\tiny\color{gray},
  keywordstyle=\color{blue},
  commentstyle=\color{black},
  stringstyle=\color{red},
  breaklines=true,
  breakatwhitespace=true,
  tabsize=3
}
\lstinputlisting{BiomassGrowthAnalysis_Amitesh.java}

\appendixsection{Line graph generator for spatial distribution analysis}
    \lstset{frame=tb,
  language=Python,
  aboveskip=3mm,
  belowskip=3mm,
  showstringspaces=false,
  columns=flexible,
  caption=Python script to generate line graphs and save them as images,
  basicstyle={\small\ttfamily},
  numbers=none,
  numberstyle=\tiny\color{gray},
  keywordstyle=\color{blue},
  commentstyle=\color{black},
  stringstyle=\color{red},
  breaklines=true,
  breakatwhitespace=true,
  tabsize=3
}
\lstinputlisting{graphGenerator.py}
    
    
    
    


