%
%  Time-stamp: "[publicabstract.tex] last modified by Scott Budge (scott) on 2011-08-09 (Tuesday, 9 August 2011) at 09:17:43 on goga"
%
%  Info: $Id$   USU
%  Revision: $Rev$
% $LastChangedDate$
% $LastChangedBy$
%

\begin{publicabstract}
% A space is needed before the text starts so that the first paragraph
% is indented properly.

In this study, we have created a simulation model which is concerned about digesting cellulose, as a major component of microalgae in a bioreactor. This model is designed to generate a computational model that simulates the process of granulation in anaerobic sludge and aims to investigate scenarios of possible granular bioaugmentation. Once a mature granule is formed, protein is used as an alternative substrate that will be supplied to a mature granule. Protein, being a main component of cyanobacteria, will promote growth and incorporation of a cell type that can degrade protein (selective pressure). The model developed in a cDynoMiCs simulation environment successfully demonstrated the process of granule formation and bioaugmentation in an Anaerobic granule.{\cite{koon}}
Bioaugmentation is a common strategy in the field of wastewater treatment, used to introduce a new metabolic capability to either aerobic or anaerobic granules. The end product of our work is a model that can visually demonstrate varying stratifications of different trophic microbial groups that will be of help for the engineers and researchers, who are operating both laboratory and industrial-scale anaerobic digesters and wish to enhance reactor performance.  
The working model that we have developed has been validated using the existing literature and lab experiments. The model successfully demonstrates granulation in a cellobiose fed system with formation of 0.63 mm mature granule in 59 days with the production of good amount of methane that could be used commercially as a green fuel. We extended this model to perform bioaugmentation by chaining different simulations. 


\end{publicabstract}


% Local Variables:
% TeX-master: "newhead"
% End:
